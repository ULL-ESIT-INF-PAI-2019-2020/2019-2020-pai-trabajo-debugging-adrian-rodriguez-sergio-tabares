%
% ---------------------------------------------------
%
% Trabajo Fin de Grado:
% Author: NombreAutor Apellido1 Apellido2 <gonzalezsuarezivan@gmail.com>
% Author: F. de Sande fsande@ull.es
% Fichero: main.tex
%
% ----------------------------------------------------
% 
\documentclass[spanish,a4paper,12pt,oneside]{extreport}
\usepackage{listingsutf8}
%\documentclass[a4paper, twoside, 12pt]{book}
\usepackage[a4paper]{geometry}
\usepackage[spanish]{babel}
\usepackage[utf8]{inputenc}
%\usepackage{lscape}
\usepackage{pdflscape}
%%%%%%%%%%%%%%%%%%%%%%%%%%%%%%%%%%%%%%%%%%%%%%%%%%%%%%%%%%%%%%%%%%%%%%%%%%%%%%%%%%%%%%%%%%%%
% Next 3+3 lines select PDF or PS output (comment as apropriate)
% To switch from PDF and PS comment/uncomment here and change Makefile
\usepackage[pdftex]{color}
\usepackage[pdftex]{graphicx}
\graphicspath{{img/}}
%\usepackage[dvips]{color}
%\usepackage[dvips]{graphicx}
\usepackage{epsfig}
%\graphicspath{{img/eps/}}
%Añadidos BulletPoint
\usepackage{floatrow}
%%%%%%%%%%%%%%%%%%%%%%%%%%%%%%%%%%%%%%%%%%%%%%%%%%%%%%%%%%%%%%%%%%%%%%%%%%%%%%%%%%%%%%%%%%%%
\usepackage{algorithmic}
\usepackage[pdftex=true,colorlinks=false,urlcolor=blue,plainpages=false,pagebackref=true,citecolor=red]{hyperref} %hiperenlaces y backcites 
\usepackage{url}
\usepackage{subcaption}
%%%%%%%%%%%%%%%%%%%%%%%%%%%%%%%%%%%%%%%%%%%%%%%%%%%%%%%%%%%%%%%%%%%%%%%%%%%%%%%%%%%%%%%%%%%
% Comandos para escribir "siempre igual"
\newcommand{\BulletP}{\texttt{PAI-X{ Tecnología de Título título título}}}
\newcommand{\ULLARP}{\texttt{PAI-X{ Tecnología de Título título título}}}

\newcommand{\ULLAR}{\texttt{PAI-X}{}}

%%% Traducimos el pseudocodigo
\renewcommand{\algorithmicwhile}{\textbf{mientras}}
\renewcommand{\algorithmicend}{\textbf{fin}}
\renewcommand{\algorithmicdo}{\textbf{hacer}}
\renewcommand{\algorithmicif}{\textbf{si}}
\renewcommand{\algorithmicthen}{\textbf{entonces}}
\renewcommand{\algorithmicrepeat}{\textbf{repetir}}
\renewcommand{\algorithmicuntil}{\textbf{hasta que}}
\renewcommand{\algorithmicelse}{\textbf{en otro caso}}
\renewcommand{\algorithmicfor}{\textbf{para}}

%%%%%%%%%%%%%%%%% Se crea un entorno para listar código fuente %%%%%%%%%%%%%%%
\newenvironment{sourcecode}
{\begin{list}{}{\setlength{\leftmargin}{1em}}\item\scriptsize\bfseries}
{\end{list}}

\newenvironment{littlesourcecode}
{\begin{list}{}{\setlength{\leftmargin}{1em}}\item\tiny\bfseries}
{\end{list}}

\newenvironment{summary}
{\par\noindent\begin{center}\textbf{Abstract}\end{center}\begin{itshape}\par\noindent}
{\end{itshape}}

\newenvironment{keywords}
{\begin{list}{}{\setlength{\leftmargin}{1em}}\item[\hskip\labelsep \bfseries Keywords:]}
{\end{list}}

\newenvironment{palabrasClave}
{\begin{list}{}{\setlength{\leftmargin}{1em}}\item[\hskip\labelsep \bfseries Palabras clave:]}
{\end{list}}


%%%%%%%%%%%%%%%%%%%%%%%%%%%%%%%%%%%%%%%%%%%%%%%%%%%%%%%%%%%%%%%%%%%%%%%%%%%%%%%
\definecolor{marron}       {rgb}{0.496, 0.203, 0.152}
\definecolor{verde-claro}  {rgb}{0.625, 0.734, 0.199}
\definecolor{oscuro}       {rgb}{0.187, 0.141, 0.285}
\definecolor{gris}     	   {rgb}{0.500, 0.500, 0.500}
\definecolor{bgd-listings} {rgb}{0.999, 0.999, 0.900}
\definecolor{gray97}{gray}{.97}
\definecolor{gray75}{gray}{.75}
\definecolor{gray45}{gray}{.45}
\definecolor{gray}{gray}{.45}
%%%%%%%%%%%%%%%%%%%%%%%%%%%%%%%%%%%%%%%%%%%%%%%%%%%%%%%%%%%%%%%%%%%%%%%%%%%%%%%%%%%%%%%%%%%%
%%% Code Listings
%\usepackage{listings} 
%\lstloadlanguages{python,C}
\definecolor{Brown}{cmyk}{0,0.81,1,0.60}
\definecolor{OliveGreen}{cmyk}{0.64,0,0.95,0.40}
\definecolor{CadetBlue}{cmyk}{0.62,0.57,0.23,0}
\definecolor{lightlightgray}{gray}{0.9}
%%%%%%%%%%%%%%%%%%%%%%%%%%%%%%%%%%%%%%%%%%%%%%%%%%%%%%%%%%%%%%%%%%%%%%%%%%%%%%%%%%%%%%%%%%%
%Evitar partir palabras al final de la línea
\hyphenpenalty=10000
\tolerance=1000
%%%%%%%%%%%%%%%%%%%%%%%%%%%%%%%%%%%%%%%%%%%%%%%%%%%%%%%%%%%%%%%%%%%%%%%%%%%%%%%%%%%%%%%%%%%%
% Para listados de código
\usepackage{listings}
\lstloadlanguages{C}

% Definiendo colores para los listados de código fuente - Univ. Deusto
\definecolor{violet}{rgb}{0.5,0,0.5}
\definecolor{lightgray}{rgb}{.9,.9,.9}
\definecolor{darkgray}{rgb}{.4,.4,.4}
\definecolor{purple}{rgb}{0.65, 0.12, 0.82}
\definecolor{navy}{rgb}{0,0,0.5}
\definecolor{hellgelb}{rgb}{1,1,0.8}
\definecolor{colKeys}{rgb}{0,0,1}
\definecolor{colIdentifier}{rgb}{0,0,0}
\definecolor{colComments}{rgb}{1,0,0}
\definecolor{colString}{rgb}{0,0.5,0}

%\lstset{morekeywords={pragma copy\_in copy\_out copy omp parallel private reduction shared hicuda loop\_partition over\_tblock over\_thread}}
\lstset{
        float=tbhp,
		    language = Java,
				morekeywords={llc,reduction_type,nc_result,
				              hicuda,global,alloc,shape,kernel,thread,loop_partition,tblock,over_tblock,over_thread,kernel_end,copyout,free,
											data,region,
											task,input,inout,output,
				              pragma,omp,parallel,reduction,private,shared,target,device,copy_in,copy_out,
				              acc,kernels,loop,copyin,copy,pcopy,pcopyin,collapse,gang,worker,independent},
				%\emph      ={omp,parallel,reduction,private,shared},
				emphstyle=\textbf,
        %basicstyle=\ttfamily\tiny,
        basicstyle=\ttfamily\scriptsize,
        identifierstyle=\color{colIdentifier},
        keywordstyle=\color{colKeys},
        stringstyle=\color{colString},
        commentstyle=\color[rgb]{0.133,0.545,0.133},
        columns=flexible,
        tabsize=4,
        frame=single,
        extendedchars=true,
        showspaces=false,
        showstringspaces=false,
        numbers=left,
        numberstyle=\tiny,
        breaklines=true,
        backgroundcolor=\color{lightlightgray},
        breakautoindent=true,
        captionpos=b
}

\lstdefinelanguage{JavaScript}{
  keywords={typeof, new, true, false, catch, function, return, null, catch, switch, var, const, let, async, await, if, in, while, do, else, case, break, from},
  ndkeywords={class, export, boolean, throw, implements, import, this},
  sensitive=false,
  comment=[l]{//},
  morecomment=[s]{/*}{*/},
  morestring=[b]',
  morestring=[b]"
}

%\renewcommand{\lstlistingname}{Listing} % Los títulos de los códigos insertados se denotan con Ejemplo...   

% Otro formato más bonito para código fuente
\newcommand{\codigofuente}[3]{%
  \lstlisting[language=#1,caption={#2}]{#3}%
}
%%%%%%%%%%%%%%%%%%%%%%%%%%%%%%%%%%%%%%%%%%%%%%%%%%%%%%%%%%%%%%%%%%%%%%%%%%%%%%%
\begin{document}
\renewcommand{\lstlistingname}{Listado}% Listing -> Listado de código
%%%%%%%%%%%%%%%%%%%%%%%%%%%%%%%%%%%%%%%%%%%%%%%%%%%%%%%%%%%%%%%%%%%%%%%%%%%%%%%
% First Page
%%%%%%%%%%%%%%%%%%%%%%%%%%%%%%%%%%%%%%%%%%%%%%%%%%%%%%%%%%%%%%%%%%%%%%%%%%%%%%%

\pagestyle{empty}
\thispagestyle{empty}


\newcommand{\HRule}{\rule{\linewidth}{1mm}}
\setlength{\parindent}{0mm}
\setlength{\parskip}{0mm}

\vspace*{\stretch{0.5}}

\begin{center}
\includegraphics[scale=1.1]{img/marca-universidad-de-la-laguna-original}\\[15mm]
{\Huge Trabajo de PAI}
\end{center}

\HRule
\begin{flushright}
        {\Huge \BulletP{}} \\[2.5mm]
        {\Large NombreAutor Apellido1 Apellido2} \\[5mm]


\end{flushright}
\HRule
\vspace*{\stretch{2}}
\begin{center}
  \Large La Laguna, \today
\end{center}

\setlength{\parindent}{5mm}

%%%%%%%%%%%%%%%%%%%%%%%%%%%%%%%%%%%%%%%%%%%%%%%%%%%%%%%%%%%%%%%%%%%%%%%%%%%%%%%
\newpage

\begin{huge}
Licencia
\end{huge}

\bigskip
%* Si quiere permitir que se compartan las adaptaciones de tu obra mientras se comparta de la misma manera
%y NO quieres permitir usos comerciales de tu obra indica:

\begin{center}
\includegraphics[scale=1.5]{img/by-nc-sa_88x31}\\[10mm]
{\Large \copyright~Esta obra está bajo una licencia de Creative Commons Reconocimiento-NoComercial-CompartirIgual 4.0 Internacional.
}
\end{center}

%%%%%%%%%%%%%%%%%%%%%%%%%%%%%%%%%%%%%%%%%%%%%%%%%%%%%%%%%%%%%%%%%%%%%%%%%%%%%%%
\newpage  %\cleardoublepage
\begin{abstract}
{\em

Este documento refleja bla, bla, bla...

}
\begin{palabrasClave}
Aplicaciones Android, Java, \textit{Cloud Computing}, Dispositivos Móviles, Programación, Realidad Aumentada, Node.js, MongoDB, Google Maps.
\end{palabrasClave}

\end{abstract}
%%%%%%%%%%%%%%%%%%%%%%%%%%%%%%%%%%%%%%%%%%%%%%%%%%%%%%%%%%%%%%%%%%%%%%%%%%%%%%%

%%%%%%%%%%%%%%%%%%%%%%%%%%%%%%%%%%%%%%%%%%%%%%%%%%%%%%%%%%%%%%%%%%%%%%%%%%%%%%%
\newpage  %\cleardoublepage
\begin{summary}

{\em
This document reflects bla, bla, bla...

}

\begin{keywords}
Application for Android, Java, Cloud Computing, Mobile Devices, Programming, Augmented Reality, Node.js, MongoDB, Google Maps.
\end{keywords}

\end{summary}
%%%%%%%%%%%%%%%%%%%%%%%%%%%%%%%%%%%%%%%%%%%%%%%%%%%%%%%%%%%%%%%%%%%%%%%%%%%%%%%

%%%%%%%%%%%%%%%%%%%%%%%%%%%%%%%%%%%%%%%%%%%%%%%%%%%%%%%%%%%%%%%%%%%%%%%%%%%%%%%
\newpage{\pagestyle{empty}}
\thispagestyle{empty}

%%%%%%%%%%%%%%%%%%%%%%%%%%%%%%%%%%%%%%%%%%%%%%%%%%%%%%%%%%%%%%%%%%%%%%%%%%%%%%%


\pagestyle{myheadings} %my head defined by markboth or markright
% No funciona bien \markboth sin "twoside" en \documentclass, pero al
% ponerlo se dan un montón de errores de underfull \vbox, con lo que no se
% ha puesto.
\markboth{NombreAutor Apellido1 Apellido2}{ULLAR}

%%%%%%%%%%%%%%%%%%%%%%%%%%%%%%%%%%%%%%%%%%%%%%%%%%%%%%%%%%%%%%%%%%%%%%%%%%%%%%%
%Numeracion en romanos
\renewcommand{\thepage}{\roman{page}}
\setcounter{page}{1}

%%%%%%%%%%%%%%%%%%%%%%%%%%%%%%%%%%%%%%%%%%%%%%%%%%%%%%%%%%%%%%%%%%%%%%%%%%%%%%%

\tableofcontents

%%%%%%%%%%%%%%%%%%%%%%%%%%%%%%%%%%%%%%%%%%%%%%%%%%%%%%%%%%%%%%%%%%%%%%%%%%%%%%%
\newpage{\pagestyle{empty}}

\listoffigures

%%%%%%%%%%%%%%%%%%%%%%%%%%%%%%%%%%%%%%%%%%%%%%%%%%%%%%%%%%%%%%%%%%%%%%%%%%%%%%%
\newpage{\pagestyle{empty}}

%\listoftables

%%%%%%%%%%%%%%%%%%%%%%%%%%%%%%%%%%%%%%%%%%%%%%%%%%%%%%%%%%%%%%%%%%%%%%%%%%%%%%%
\newpage{\pagestyle{empty}}

%%%%%%%%%%%%%%%%%%%%%%%%%%%%%%%%%%%%%%%%%%%%%%%%%%%%%%%%%%%%%%%%%%%%%%%%%%%%%%%
%Numeracion a partir del capitulo I
\renewcommand{\thepage}{\arabic{page}}
\setcounter{page}{1}


% ==========================================================
% --------               Capítulos                ----------
% --------    Estan en el directorio capitulos/   ----------
% ==========================================================
% ---------------------------------------------------
% Universidad de La Laguna
% Escuela Superior de Ingeniería y Tecnología
% Grado en Ingeniería Informática
% Programación de Aplicaciones Interactivas (PAI)
%
% Trabajo PAI
% Autor: <aluXXXXXXXXXX@ull.edu.es>
% Date:
% Capítulo: 
% Fichero: 
% ----------------------------------------------------
 
Preface

% ---------------------------------------------------
% Universidad de La Laguna
% Escuela Superior de Ingeniería y Tecnología
% Grado en Ingeniería Informática
% Programación de Aplicaciones Interactivas (PAI)
%
% Trabajo PAI
% Autor: <aluXXXXXXXXXX@ull.edu.es>
% Date:
% Capítulo: 
% Fichero: 
% ----------------------------------------------------
 

\chapter{Objetivos} \label{chap:Objetivos}  

Este documento bla, bla, bla...

Este Trabajo tiene los siguientes objetivos :
	
\begin{enumerate}
\item Por un lado se pretende 

\item Otro objetivo presente es que 

\item A su vez, se busca que 

\item También se pretende que utilizando \textit{Github} \cite{URL::Github} y de edición de textos técnicos utilizando \textit{LaTeX}  \cite{URL::LaTeX}.

\item   Por último, 
\end{enumerate} 

% ---------------------------------------------------
% Universidad de La Laguna
% Escuela Superior de Ingeniería y Tecnología
% Grado en Ingeniería Informática
% Programación de Aplicaciones Interactivas (PAI)
%
% Trabajo PAI
% Autor: <aluXXXXXXXXXX@ull.edu.es>
% Date:
% Capítulo: 
% Fichero: 
% ----------------------------------------------------
 
\cleardoublepage
\chapter{Herramientas y Tecnologías} \label{chap:Tecnologias} 

Este capítulo tiene como objetivo presentar ...

\section{Herramientas de Desarrollo}

\subsection{Android Studio}

\begin{figure}[h]
    \centering
    \includegraphics[width=0.6\linewidth]{marca-universidad-de-la-laguna-original}
    \caption{Android Studio, un IDE flexible e intuitivo.}
    \label{fig:androidstudio}
\end{figure}

\subsection{LaTex}

LaTeX \cite{URL::LaTeX} es un sistema de composición de textos, 

\subsection{Github}

GitHub \cite{URL::Github} es una plataforma de desarrollo colaborativo para alojar proyectos que utiliza el sistema de control de versiones Git. 
GitHub fue escrito en Ruby on Rails. 
El código se almacena de forma pública, aunque también se puede hacer de forma privada, creando una cuenta de pago.

Para instalar GitHub en Linux se utiliza el siguiente comando:
\begin{lstlisting}
    sudo apt install git-all
\end{lstlisting}


% ---------------------------------------------------
% Universidad de La Laguna
% Escuela Superior de Ingeniería y Tecnología
% Grado en Ingeniería Informática
% Programación de Aplicaciones Interactivas (PAI)
%
% Trabajo PAI
% Autor: <aluXXXXXXXXXX@ull.edu.es>
% Date:
% Capítulo: 
% Fichero: 
% ----------------------------------------------------
 
\chapter{Entornos universitarios } \label{chap:RAEntornosUniversitarios}  

En este capítulo se expondrán los usos y ventajas de ...

\begin{figure}[h]
    \centering
    \includegraphics[width=0.6\linewidth]{marca-universidad-de-la-laguna-original}
    \caption{Construcción de un automóvil de carreras.}
    \label{fig:xyz}
\end{figure}    

% ---------------------------------------------------
% Universidad de La Laguna
% Escuela Superior de Ingeniería y Tecnología
% Grado en Ingeniería Informática
% Programación de Aplicaciones Interactivas (PAI)
%
% Trabajo PAI
% Autor: <aluXXXXXXXXXX@ull.edu.es>
% Date:
% Capítulo: 
% Fichero: 
% ----------------------------------------------------
 

\lstset{stringstyle=\color{purple}}
\chapter{La aplicación} \label{chap:LaAplicacion} 

En este capítulo se explicará en detalle la aplicación 

Los requisitos principales  son:
\begin{itemize}
    \item La aplicación se 
    \item Se implementarán técnicas de 
    \item Las instalaciones de la ULL, 
\end{itemize}

\subsection{Especificación detallada de los requisitos} 

%\begin{figure}[h]
%    \hspace*{\fill}%
%    \begin{subfigure}[h]{0.35\linewidth}
%    \includegraphics[width=\linewidth]{by-nc-sa_88x31}
%    \caption{Configuración.}
%    \label{fig:settingsApp}
%    \end{subfigure}
%    \hfill%
%    \begin{subfigure}[h]{0.35\linewidth}
%    \includegraphics[width=\linewidth]{marca-universidad-de-la-laguna-original}
%    \caption{Información.}
%    \label{fig:infoApp}
%    \end{subfigure}%
%    \caption{Ventanas de \textit{Configuración} e \textit{Información} }.}
%    \hspace*{\fill}%
%\end{figure}


\section{Inicio de \ULLAR{}} \label{chap:StartApplication} 

\begin{lstlisting}[ label={lst:SHA1}]
     keytool -list -v -alias androiddebugkey -keystore ~/.android/debug.keystore
\end{lstlisting} 

\begin{lstlisting}[caption={Fichero \texttt{build.gradle} del proyecto, dependencias para utilizar los Servicios de Google.}, label={lst:googleSd}]
...
buildscript{
    dependencies {
        // Dependencias de los Servicios de Google
        classpath 'com.google.gms:google-services:4.0.0'
    }    
} 
...
\end{lstlisting}
 

 
\begin{figure}[h]
    \centering
    \includegraphics[width=0.38\linewidth]{by-nc-sa_88x31}
    \caption{Disposición de los ejes de un dispositivo Android.}
    \label{fig:xyz}
\end{figure}    

\begin{minipage}{\linewidth}
\begin{lstlisting}[caption={Código que se ejecuta cada vez que se registra un cambio en el sensor que calcula la orientación.}, label={lst:orientacionL}]
    // Se escuchan los cambios en el sensor y se hacen los calculos
    public void onSensorChanged(SensorEvent event) {
        // Valor del sensor en grados
        double radians = event.values[0]; 
        // Se convierte a radianes
        radians = Math.toRadians(radians);
        // Se obtiene la ultima posicion registrada del GPS
        LatLng lastPosition = getCurrentPos();
        if (auxpos != null) { // Si la posicion no es nula
            // Se le pregunta al objeto de la clase ``Navigation'' las instalaciones 
            // que se encuentran en esa direccion
            allResultsSites = navULL.whatCanSee(lastPosition, radians);
        }
        // Si se obtiene al menos un resultado
        if (allResultsSites != null) {
            // Se obtiene la instalacion mas cercana, el indice 0 corresponde a la mas cercana
            nearSiteResult = allResultsSites.get(0);
            ... // Se muestra su informacion por pantalla para que usuario conozca la instalacion  
                // a la que se encuentra apuntando y un boton que lanza una con una ficha de 
                // informacion sobre esta
            if(allResultsSites.size() > 2) {
                ... // Si se obtiene mas de una instalacion se muestra al usuario el boton
                    // que indica el numero de instalaciones que se encuentran en la misma
                    // direccion y lanza una ventana con una lista de estas
            }
        } else { ... }
    } 
\end{lstlisting}
\end{minipage}

A continuación, se instalarán los paquetes necesarios para funcionamiento del servidor Node.js. Estos son:

\begin{itemize}
    \item \textbf{ExpressJS}: Express es una infraestructura de aplicaciones web Node.js mínima y flexible que proporciona un conjunto sólido de características para las aplicaciones web y móviles.
    \item \textbf{mongoose}: Mongoose es una librería para trabajar MongoDB y Node.js.
    \item \textbf{bodyparser}: Se necesitará para manejar las peticiones de JSON.
    \item \textbf{node-restful}: Sirve para manejar las peticiones recibidas del servidor y conectarse con una base de datos de MongoDB.
\end{itemize}

Con un comando se instalarán todos los paquetes y se guardarán las dependencias utilizadas en el fichero \texttt{package.json}:

\lstinputlisting[language=JavaScript, caption={Fichero \texttt{computePI.js}.}, label={code:ullSites.js},]{listings/computePI.js} %% LISTING


% ---------------------------------------------------
% Universidad de La Laguna
% Escuela Superior de Ingeniería y Tecnología
% Grado en Ingeniería Informática
% Programación de Aplicaciones Interactivas (PAI)
%
% Trabajo PAI
% Autor: <aluXXXXXXXXXX@ull.edu.es>
% Date:
% Capítulo: 
% Fichero: 
% ----------------------------------------------------
 
\chapter{Conclusiones y futuras líneas de Trabajo} \label{chap:Conclusiones} 

En este capítulo se presentarán las conclusiones 

\section{Conclusiones}
 
Actualmente la bla, bla, bla...

  
\addcontentsline{toc}{chapter}{Bibliografía}
% \bibliographystyle{plain}
\bibliographystyle{ieeetr}
  % \bibliographystyle{bmc_article} 
\renewcommand{\bibname}{Bibliografía}   %  Para que no aparezca Índice de figuras
\bibliography{bibliografia}

%%%%%%%%%%%%%%%%%%%%%%%%%%%%%%%%%%%%%%%%%%%%%%%%%%%%%%%%%%%%%%%%%%%%%%%%%%%%%%%
 
\end{document}
