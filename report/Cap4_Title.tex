% ---------------------------------------------------
% Universidad de La Laguna
% Escuela Superior de Ingeniería y Tecnología
% Grado en Ingeniería Informática
% Programación de Aplicaciones Interactivas (PAI)
%
% Trabajo PAI
% Autor: <aluXXXXXXXXXX@ull.edu.es>
% Date:
% Capítulo: 
% Fichero: 
% ----------------------------------------------------
 

\lstset{stringstyle=\color{purple}}
\chapter{La aplicación} \label{chap:LaAplicacion} 

En este capítulo se explicará en detalle la aplicación 

Los requisitos principales  son:
\begin{itemize}
    \item La aplicación se 
    \item Se implementarán técnicas de 
    \item Las instalaciones de la ULL, 
\end{itemize}

\subsection{Especificación detallada de los requisitos} 

%\begin{figure}[h]
%    \hspace*{\fill}%
%    \begin{subfigure}[h]{0.35\linewidth}
%    \includegraphics[width=\linewidth]{by-nc-sa_88x31}
%    \caption{Configuración.}
%    \label{fig:settingsApp}
%    \end{subfigure}
%    \hfill%
%    \begin{subfigure}[h]{0.35\linewidth}
%    \includegraphics[width=\linewidth]{marca-universidad-de-la-laguna-original}
%    \caption{Información.}
%    \label{fig:infoApp}
%    \end{subfigure}%
%    \caption{Ventanas de \textit{Configuración} e \textit{Información} }.}
%    \hspace*{\fill}%
%\end{figure}


\section{Inicio de \ULLAR{}} \label{chap:StartApplication} 

\begin{lstlisting}[ label={lst:SHA1}]
     keytool -list -v -alias androiddebugkey -keystore ~/.android/debug.keystore
\end{lstlisting} 

\begin{lstlisting}[caption={Fichero \texttt{build.gradle} del proyecto, dependencias para utilizar los Servicios de Google.}, label={lst:googleSd}]
...
buildscript{
    dependencies {
        // Dependencias de los Servicios de Google
        classpath 'com.google.gms:google-services:4.0.0'
    }    
} 
...
\end{lstlisting}
 

 
\begin{figure}[h]
    \centering
    \includegraphics[width=0.38\linewidth]{by-nc-sa_88x31}
    \caption{Disposición de los ejes de un dispositivo Android.}
    \label{fig:xyz}
\end{figure}    

\begin{minipage}{\linewidth}
\begin{lstlisting}[caption={Código que se ejecuta cada vez que se registra un cambio en el sensor que calcula la orientación.}, label={lst:orientacionL}]
    // Se escuchan los cambios en el sensor y se hacen los calculos
    public void onSensorChanged(SensorEvent event) {
        // Valor del sensor en grados
        double radians = event.values[0]; 
        // Se convierte a radianes
        radians = Math.toRadians(radians);
        // Se obtiene la ultima posicion registrada del GPS
        LatLng lastPosition = getCurrentPos();
        if (auxpos != null) { // Si la posicion no es nula
            // Se le pregunta al objeto de la clase ``Navigation'' las instalaciones 
            // que se encuentran en esa direccion
            allResultsSites = navULL.whatCanSee(lastPosition, radians);
        }
        // Si se obtiene al menos un resultado
        if (allResultsSites != null) {
            // Se obtiene la instalacion mas cercana, el indice 0 corresponde a la mas cercana
            nearSiteResult = allResultsSites.get(0);
            ... // Se muestra su informacion por pantalla para que usuario conozca la instalacion  
                // a la que se encuentra apuntando y un boton que lanza una con una ficha de 
                // informacion sobre esta
            if(allResultsSites.size() > 2) {
                ... // Si se obtiene mas de una instalacion se muestra al usuario el boton
                    // que indica el numero de instalaciones que se encuentran en la misma
                    // direccion y lanza una ventana con una lista de estas
            }
        } else { ... }
    } 
\end{lstlisting}
\end{minipage}

A continuación, se instalarán los paquetes necesarios para funcionamiento del servidor Node.js. Estos son:

\begin{itemize}
    \item \textbf{ExpressJS}: Express es una infraestructura de aplicaciones web Node.js mínima y flexible que proporciona un conjunto sólido de características para las aplicaciones web y móviles.
    \item \textbf{mongoose}: Mongoose es una librería para trabajar MongoDB y Node.js.
    \item \textbf{bodyparser}: Se necesitará para manejar las peticiones de JSON.
    \item \textbf{node-restful}: Sirve para manejar las peticiones recibidas del servidor y conectarse con una base de datos de MongoDB.
\end{itemize}

Con un comando se instalarán todos los paquetes y se guardarán las dependencias utilizadas en el fichero \texttt{package.json}:

\lstinputlisting[language=JavaScript, caption={Fichero \texttt{computePI.js}.}, label={code:ullSites.js},]{listings/computePI.js} %% LISTING

